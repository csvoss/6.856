\section{Conclusion}
We have demonstrated Yao's principle to give lower bounds on the competitiveness of algorithms for a couple of online games, which is an upper bound on their effectiveness. When these bounds match the competitiveness of a particular algorithm, the principle can be used to show the tightness of the algorithm.

Also, there is a dual between using an algorithm to demonstrate a lower bound on the solvability of a problem (alternatively, an upper bound on the hardness of a problem), and using a series of inputs to demonstrate an upper bound on the solvability of a problem (alternatively, an lower bound on the hardness). As more work is done on a particular problem, the gap between the two shrink until a provably optimal algorithm is found.

Yao's principle is applicable to many types of costs when analyzing problems, including running time, and, as we saw here, competitiveness of algorithms in partial information scenarios.
