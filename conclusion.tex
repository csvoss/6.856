\section{Conclusion}
We have demonstrated Yao's principle to give lower bounds on the competitiveness of algorithms for a couple of online games, which is an upper bound on their effectiveness. When these bounds match the competitiveness of a particular algorithm, the principle can be used to show the tightness of the algorithm.

There is a dual between using an algorithm to demonstrate a lower bound on the solvability of a problem (alternatively, an upper bound on the hardness of a problem), and using a series of inputs to demonstrate an upper bound on the solvability of a problem (alternatively, an lower bound on the hardness). As more work is done on a particular problem, the gap between the two shrink until a provably optimal algorithm is found.

Online problems, such as the removable online knapsack problem and the and the online coloring co-interval graphs problem, all deal with situations of \emph{partial information}: Quantum non-local games are also situations of partial information: the players must form optimal strategies despite receiving input at random from an adversary. While Yao's principle is applicable to many types of costs when analyzing problems, including running time, as we saw here it can also be applied to analyzing the competitiveness of algorithms in partial information scenarios.
