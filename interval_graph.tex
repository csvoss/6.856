\documentclass{article}
\usepackage[utf8]{inputenc}
\usepackage{fancyvrb}
\usepackage{bera}
\usepackage{amsmath}

\title{6.856 Final Project}
\author{ }
\date{May 2015}

\begin{document}

\maketitle

\section{Online Coloring Co-Interval Graphs}
This section studies the online problem of coloring co-interval graphs. \\This problem is set up as follows:
\\Input: A set of intervals on the real line in arbitrary order
\\Output: A color matched to each interval satisfying the condition that any interval $I_k$ must be assigned a color that's different from the color assigned to any $I_j$ such that $j < k$ and intervals $j$ and $k$ don't intersect.
\\Since this is an online problem, the algorithm must assign colors as it goes, without knowing the complete set of intervals. That is, the algorithm will only receive information about $I_{j+1}$ after it has assigned colors to all intervals $I_k$ where $k \leq j$.
\\First, we present the First-Fit approximation algorithm to solve this problem, which has a competitive ratio of at most 2. Then, using Yao's Principle, we can prove that there exist no randomized algorithm that achieves a competitive ratio that's better than $\frac{3}{2}$.
\subsection{Upper Bound}
To solve the problem of assigning intervals different colors, the set of intervals is first constructed as a graph. That is, the input to our problem becomes:
\\$V$ ::== Set of intervals where $v_i$ represents the $i$'th interval
\\$E$ ::== Set of edges where $(u,v) \in E$ iff interval $u$ and interval $v$ are disjoint
\\Note that any vertices that aren't connected by an edge can be colored in the same color since they overlap. Thus, in such a graph, our problem is reduced to finding the minimum number of colors such that no connected vertices are the same color.
\\The First-Fit algorithm is the simplest solution to this problem and does the following:
\begin{verbatim}
L = sorted list of colors
for a new vertex v:
    assign v the smallest color that doesn't cause a collision
\end{verbatim}
Essentially, the algorithm maintains a list of colors that can be used. These colors are sorted in an arbitrary order. The first vertex is simply assigned the smallest color. For any future vertex $v$, the algorithm scans the list. We can define a collision between vertices $u$ and $v$ at color $k$ if $u$ is colored $k$ and $v$ is connected to $u$ by an edge in $E$. If a collision occurs where $L[i] = k$, then we increment $i$. We assign $v$ the color at $L[j]$ where $j$ doesn't cause any collisions and try to minimize $j$.
\\It has been shown by Gyarfas and Lehel that First-Fit uses at most $2\omega+1$ colors on any co-chordal graph, which is a class that includes co-interval graphs. It has also been shown by Kierstead and Qin that no online deterministic algorithm can beat the 2-competitive ratio on co-interval graphs.
\\In the next section, we look at using randomization in co-interval graphs and lower bound the best competitive ratio that can be obtained.
\subsection{Lower Bound}
Yao's minimax principle proves that the expected competitive ratio on an optimal deterministic algorithm on the worst-case input is a lower bound on the expected competitive ratio of all randomized algorithms. That is, if there exists a set of inputs such that all optimal deterministic algorithms are at least $\alpha$-competitive, then $\alpha$ is the best ratio that can be obtained by using randomization.
\\To define such a set of inputs, essentially, we need to find a set of intervals that satisfy the condition that any deterministic algorithm is at most $\frac{3}{2}$-competitive. The idea behind constructing such a set of intervals is that any deterministic algorithm will make the same decision on any specific interval given the same information, regardless of what the future intervals are.
\\The intervals defined by Zarrabi-Zadeh are: 
\begin{Verbatim}[commandchars=\\\{\},codes={\catcode`$=3\catcode`_=8}]
For k $\geq$ 1:
    For all 1 $\leq i \leq k$:
        $a_i$ = [3$i$-3, 3$i$-2]
        $b_i$ = [3$i$+1, 3$i$+2]
        $c_i$ = [3$i$+2, +$\infty$]
\end{Verbatim}
We can then define a block as:
$B_i$ = \begin{cases} [$a_1, $b_1$] &\mbox{if } i \equiv 1 \\ 
[$a_i$, $b_i$, $c_i$] & \mbox{if } i \in [2,k] \\ [$a_k$, $b_k$] &\mbox{if } i \equiv k \end{cases}
\\From this set of blocks, we can choose a set of $k$ input sequences, simply by defining our input sequence $X_i$ as the concatenation of blocks $B_1$, $B_2$... $B_i$, in order.
\\It can be shown that the minimum number of colors required to color $X_i$ without any edges sharing a color is $i$.
\\To use Yao's Principle on this input distribution, we need to show that if we set up a probability distribution over all possible $X_i$ that the expected competitive ratio will be at least $\frac{3}{2}$. The optimal deterministic algorithm that we will examine per each $X_i$ input is First-Fit - as previously shown, this algorithm uses $2\omega-1$ colors on each $X_i$. In this case, the chromatic number $\omega=i$, so, the expected number of colors for a given $X_i$ is exactly $2i-1$.

\subsection{Remarks}

\end{document}
