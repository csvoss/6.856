\section{Yao's Principle in Quantum Information}

Yao's principle isn't just used for proving worst-case bounds on the competitiveness of algorithms. Another application is found in the study of certain games in quantum information, permitting proof of worst-case bounds for strategies in these games. In this section, we will examine how Yao's minimax principle may be used for this purpose, as demonstrated in \emph{Worst Case Analysis of Non-local Games} by Ambainis et al. \cite{ABBSSV}.

\subsection{Quantum Preliminaries}

Consider a game between two players and an adversary, where the two players receive random input from the adversary and must each return outputs jointly satisfying a condition without communicating with each other during the game. Analysis under classical physics conditions and quantum physics conditions has demonstrated quantum information allows for better strategies.

An example of such a game is the CHSH (Clauser-Horne-Shimonyi-Holt) \cite{CHSH} game:

\begin{itemize}
\item{The two players may collude to set up a strategy beforehand, but may not communicate afterwards.}
\item{The adversary chooses a random $\vec{x}=(x_1, x_2) \in X=\{(0,0), (0,1), (1,0), (1,1)\}$. The random choice is uniform, with $\frac14$ probability each.}
\item{The players 1 and 2 receive the values $x_1$ and $x_2$, respectively.}
\item{In the quantum setting, the players 1 and 2 may also possess an entangled 2-part quantum state $\ket{\psi}$, which they may measure.}
\item{Using their strategy, the players 1 and 2 output the values $a_1(x_1)$ and $a_2(x_2)$, respectively.}
\item{The players win if $a_1 \oplus a_2 = x_1 \wedge x_2$.}
\end{itemize}

Both under classical physics and under any ``local hidden variable theory'' -- explanation of quantum phenomena as being due to extra local hidden variables -- the best that the players can achieve is a win probability of 0.75. If the players are allowed to share entangled quantum bits, however, then they can win more often, with a win probability of $\frac12 + \frac1{2\sqrt2} = 0.8535\ldots$ \cite{ABBSSV,CHSH}.

The CHSH game was designed as a testable experiment, and experiments have since confirmed that the universe is consistent with quantum entanglement, and not consistent with local hidden variable theories \cite{aspect}.

In \emph{Worst Case Analysis of Non-local Games} by Ambainis et al., the authors examine generalized games beyond the simple CHSH game, and analyze the worst-case performance of strategies in both classical and quantum environments \cite{ABBSSV}.

\subsection{Generalized CHSH}

The CHSH game is played between two players and an adversary. We generalize the game from two players to $n$ players. Ambainis et al.~call this the \emph{n-party AND game} \cite{ABBSSV}.

%% \begin{itemize}
%% \item{The $n$ players may collude to set up a strategy beforehand, but may not communicate afterwards.}
%% \item{The adversary chooses a random $\vec{x}=(x_1, ..., x_n) \in X_1 \times ... \times X_n$, for some $X_1, ..., X_n$. The random choice is governed by a probability distribution $\pi$; each $\vec{x}$ is chosen with probability $\pi(\vec{x})$.}
%% \item{The players $1, ..., n$ receive the values $x_1, ..., x_n$, respectively.}
%% \item{In the quantum setting, the players may also possess an entangled $n$-part quantum state $\ket{\psi}$, which they may measure.}
%% \item{Using their strategy, the players $1, ..., n$ output the values $\vec{a} = a_1(x_1), ..., a_n(x_n)$, respectively.}
%% \item{The players win if $\bigoplus \vec{a} = \bigwedge \vec{x}$. Let the indicator variable $\texttt{win}(\vec{a}|\vec{x}) = 1$ if they win, $-1$ otherwise.}
%% \end{itemize}

%% The authors note that this game has not been studied before because of the simplicity of its strategies in the classical setting: by outputting $a_i=0$, the players are guaranteed to win unless the adversary happened to pick $x_i=1$. Therefore, against an adversary that always chooses $\vec{x}$ uniformly, the players can win with probability $1-2^n$.

%% However, this game becomes more interesting when we consider worst-case bounds: if the adversary is allowed to choose $\pi$, the optimal strategy for the players becomes less trivial. The authors show that $\lim_{n \rightarrow \infty}Pr(win) = 2/3$ in both the classical case and the quantum case. 

%% TODO: transition here

%% \textbf{Theorem:} TODO

%% \begin{proof}
%% In previous results [[cite Ambainis 2010]], the authors demonstrate that the value of [...TODO].
%% \end{proof}

% \subsection{Fully Generalized Games}

Furthermore, though the n-party AND game and the CHSH game share the same win condition, $\bigoplus \vec{a} = \bigwedge \vec{x}$, we can generalize to arbitrary win conditions.

\begin{itemize}
\item{The $n$ players may collude and strategize beforehand, but may not communicate afterwards.}
\item{The adversary chooses a random $\vec{x}=(x_1, ..., x_n) \in X_1 \times ... \times X_n$, for some $X_1, ..., X_n$. The random choice is governed by a probability distribution $\pi$; each $\vec{x}$ is chosen with probability $\pi(\vec{x})$.}
\item{The players $1, ..., n$ receive the values $x_1, ..., x_n$, respectively.}
\item{In the quantum setting, the players may also possess an entangled $n$-part quantum state $\ket{\psi}$, which they may measure.}
\item{Using their strategy, the players $1, ..., n$ output the values $\vec{a} = a_1, ..., a_n$, respectively.}
\item{The players win if some proposition $\texttt{win}(\vec{a} | \vec{x})$ is true. Let the indicator variable $\texttt{win}(\vec{a}|\vec{x}) = 1$ if they win, $-1$ otherwise.}
\end{itemize}

For any game $G$, let $\omega^\pi(G) \equiv Pr(Win) - Pr(Lose)$ when $\pi$ is a probability distribution according to which the adversary chooses $\vec{x}$. Let $\omega(G)$ be the worst case $\omega^\pi(G)$: the minimum value of $\omega^\pi(G)$ over distributions $\pi$ that the adversary can choose from. Let $\omega_q$ and $\omega_c$ be $\omega$ in the quantum and classical cases, respectively. We defined the indicator variable $\texttt{win}(\vec{a}|\vec{x})$ such that $Pr(Win) - Pr(Lose) = \Expected \texttt{win}(\vec{a}|\vec{x})$. Therefore, we can more formally define $\omega(G)$ and $\omega^\pi(G)$ as follows:

$$\omega^\pi(G) \equiv \max_{\vec{a}} \Expected_{\vec{x} \in \pi} \texttt{win}(\vec{a}|\vec{x})$$

$$\omega(G) \equiv \max_{\vec{a}} \min_\pi \Expected_{\vec{x} \in \pi} \texttt{win}(\vec{a}|\vec{x})$$

Recent research in quantum information has been about finding games which exemplify a difference between the quantum and classical settings: seeking to maximize the ratio $\frac{\omega_q^\pi(G)}{\omega_c^\pi(G)}$ \cite{ABBSSV}. Interestingly, for every possible proposition $\texttt{win}$, the separation ratio $\frac{\omega_q(G)}{\omega_c(G)}$ when the adversary chooses the worst distribution $\pi$ \emph{separately} for the classical and quantum case, is at most the maximum value of $\frac{\omega_q^\pi(G)}{\omega_c^\pi(G)}$ over a fixed distribution $\pi$. This proof uses Yao's minimax principle.

\begin{theorem}
	$\frac{\omega_q(G)}{\omega_c(G)} \leq \max_\pi \frac{\omega_q^\pi(G)}{\omega_c^\pi(G)}$
\end{theorem}

\begin{proof}

As with the earlier applications of Yao's minimax principle, the adversary chooses a strategy over a distribution of inputs $\vec{x}$ to pass to the players, and the players may still use randomized algorithms as strategies. However, instead of letting the cost be \emph{running time}, the cost is the \emph{probability of winning}. The adversary attempts to minimize cost and the two players' joint algorithms attempt to maximize cost. 

Recall Yao's minimax principle: Instead of ``any (including the best) randomized algorithm when run on its own worst-case input can perform no better than the best deterministic algorithm when run on any (including the worst) specific distribution of inputs,'' we have \emph{the players' best randomized algorithms when run on their worst-case input can perform no better than the players' best deterministic algorithms when run on the worst distribution $\pi$ of inputs from the adversary}. 

The players' best deterministic algorithm against any the worst distribution $\pi$ over random inputs from the adversary is $\displaystyle\max_{\vec{a}} \min_{\pi} \Expected_{\vec{x} \in pi} \texttt{win}(\vec{a}|\vec{x})$, which is defined to be $\omega_c(G)$. The players' best randomized algorithm against the worst input is $\displaystyle\min_\pi \max_{\vec{a}} \Expected_{\vec{x} \in pi} \texttt{win}(\vec{a}|\vec{x})$, which is $\displaystyle\min_\pi \omega^\pi_c(G)$.

This gives us a bound on the worst-case win probability under classical conditions: $\displaystyle\omega_c(G) \geq \min_{\pi} \omega_c^\pi(G)$. Suppose $\pi$ is the probability distribution that achieves this minimum; then $\omega_q^\pi(G) \geq \omega_q(G)$, because fixing $\pi$ can only allow the players to achieve a better win probability than their worst-case $\omega_q(G)$.

\end{proof}

%TODO: the following citations

%Ambainis, A., A. Ba\v{c}kurs, K. Balodis, A. \v{S}ku\v{s}kovniks, J. Smotrovs, M. Virza: Worst Case Analysis of Non-local Games. P. van Emde Boas et al. (Eds.): SOFSEM 2013, LNCS 7741, pp. 121-132, 2013. \url{http://arxiv.org/abs/1112.2856}

%Ambainis, A., Kravchenko, D., Nahimovs, N., Rivosh, A.: Nonlocal Quantum XOR Games for Large Number of Players. In: Kratochv\a'il, J., Li, A., Fiala, J., Kolman, P. (eds.) TAMC 2010. LNCS, vol. 6108, pp. 72-83. Springer, Heidelberg (2010).

%Clauser, J., M. Horne, A. Shimony, R. Holt: Proposed experiment to test local
%hidden-variable theories. Physical Review Letters 23, 880 (1969). \url{http://journals.aps.org/prl/abstract/10.1103/PhysRevLett.23.880}

%Aspect, A., P. Grangier, G. Roger: Experimental Tests of Realistic Local Theories via Bell's Theorem. Physical Review Letters 47, 460 (1981). \url{http://journals.aps.org/prl/abstract/10.1103/PhysRevLett.47.460}.

%Yao, A.: Probabilistic computations: Toward a unified measure of complexity. Proceedings of the 18th IEEE Symposium on Foundations of Computer Science (FOCS), pp. 222-227 (1977).
