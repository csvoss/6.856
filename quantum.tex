\section{Yao's Principle in Quantum Information}

Yao's principle isn't just used for proving worst-case bounds on the running time of algorithms. Another area where Yao's principle has found an application is in the study of quantum information, permitting proof of worst-case bounds for strategies in certain quantum information ``games''. In this section, we will examine how Yao's minimax principle may be used for this purpose, as demonstrated in \emph{Worst Case Analysis of Non-local Games} by Ambainis et al.


\subsection{Quantum Preliminaries}

Consider setting up a ``game'' between two players and an adversary, where the two players receive random input from the adversary and must each return specific outputs based on those inputs without communicating with each other. Analyzing the best strategies for these games, under the assumptions of both classical physics and quantum physics, can be used to demonstrate that better strategies exist if the players are permitted access to quantum information.

An example of such a game is the CHSH (Clauser-Horne-Shimonyi-Holt) game:

\begin{itemize}
\item{The two players may collude to set up a strategy beforehand, but may not communicate afterwards.}
\item{The adversary chooses a random $\vec{x}=(x_1, x_2) \in X=\{(0,0), (0,1), (1,0), (1,1)\}$. The random choice is uniform, with $\frac14$ probability each.}
\item{The players 1 and 2 receive the values $x_1$ and $x_2$, respectively.}
\item{In the quantum setting, the players 1 and 2 may also possess an entangled 2-part quantum state $\ket{\psi}$, which they may measure.}
\item{Using their strategy, the players 1 and 2 output the values $a_1$ and $a_2$, respectively.}
\item{The players win if $a_1 \oplus a_2 = x_1 \wedge x_2$.}
\end{itemize}

Both under classical physics and under any ``local hidden variable theory'' -- explanation of quantum phenomena as being due to extra local hidden variables -- the best that the players can achieve is a win probability of 0.75. If the players are allowed to share entangled quantum bits, however, then they can win more often: $\frac12 + \frac1{2\sqrt(2)} = 0.8535$... becomes the best win probability that the players can achieve. [[cite Clauser]] [[cite Ambainis 2013]]

The CHSH game was designed so as to be an actual realizable experiment, and experiments have since confirmed that the way the universe works is consistent with quantum entanglement, and not consistent with local hidden variable theories. [[cite Aspect]]

In \emph{Worst Case Analysis of Non-local Games} by Ambainis et al., the authors examine generalized games beyond the simple CHSH game, and analyze the worst-case performance of strategies in both classical and quantum physics. [[cite Ambainis 2013]]


\subsection{Generalized CHSH}

The CHSH game is played between two players and an adversary; one simple generalization that can be made is to extend the game from two players to $n$ players. Ambainis et al.~call this the \emph{n-party AND game}.

\begin{itemize}
\item{The $n$ players may collude to set up a strategy beforehand, but may not communicate afterwards.}
\item{The adversary chooses a random $\vec{x}=(x_1, ..., x_n) \in X_1 \times ... \times X_n$, for some $X_1, ..., X_n$. The random choice is governed by a probability distribution $\pi$; each $\vec{x}$ is chosen with probability $\pi(\vec{x})$.}
\item{The players $1, ..., n$ receive the values $x_1, ..., x_n$, respectively.}
\item{In the quantum setting, the players may also possess an entangled $n$-part quantum state $\ket{\psi}$, which they may measure.}
\item{Using their strategy, the players $1, ..., n$ output the values $\vec{a} = a_1, ..., a_n$, respectively.}
\item{The players win if $\bigoplus \vec{a} = \bigwedge \vec{x}$.}
\end{itemize}

The authors note that this game has not been studied before because of its triviality in the classical setting: by outputting $a_i=0$, the players are guaranteed to win unless all $x_i=1$. Against an adversary that chooses $\vec{x}$ uniformly, the players can win with probability $1-2^n$.

However, this game becomes more interesting when we consider worst-case bounds: if the adversary is allowed to choose $\pi$, the optimal strategy for the players becomes less trivial.

The authors show that $\lim_{n \rightarrow \infty}Pr(win) = 2/3$ in both the classical case and the quantum case. 

[[cite Ambainis 2010]]


\subsection{Fully Generalized Games}

\begin{itemize}
\item{The $n$ players may collude to set up a strategy beforehand, but may not communicate afterwards.}
\item{The adversary chooses a random $\vec{x}=(x_1, ..., x_n) \in X_1 \times ... \times X_n$, for some $X_1, ..., X_n$. The random choice is governed by a probability distribution $\pi$; each $\vec{x}$ is chosen with probability $\pi(\vec{x})$.}
\item{The players $1, ..., n$ receive the values $x_1, ..., x_n$, respectively.}
\item{In the quantum setting, the players may also possess an entangled $n$-part quantum state $\ket{\psi}$, which they may measure.}
\item{Using their strategy, the players $1, ..., n$ output the values $\vec{a} = a_1, ..., a_n$, respectively.}
\item{The players win if some proposition $Win(\vec{a} | \vec{x})$ is true.}
\end{itemize}




To-do: the following citations

Ambainis, A., A. Ba\v{c}kurs, K. Balodis, A. \v{S}ku\v{s}kovniks, J. Smotrovs, M. Virza: Worst Case Analysis of Non-local Games. P. van Emde Boas et al. (Eds.): SOFSEM 2013, LNCS 7741, pp. 121-132, 2013. \url{http://arxiv.org/abs/1112.2856}

Ambainis, A., Kravchenko, D., Nahimovs, N., Rivosh, A.: Nonlocal Quantum XOR Games for Large Number of Players. In: Kratochv\a'il, J., Li, A., Fiala, J., Kolman, P. (eds.) TAMC 2010. LNCS, vol. 6108, pp. 72-83. Springer, Heidelberg (2010).

Clauser, J., M. Horne, A. Shimony, R. Holt: Proposed experiment to test local
hidden-variable theories. Physical Review Letters 23, 880 (1969). \url{http://journals.aps.org/prl/abstract/10.1103/PhysRevLett.23.880}

Aspect, A., P. Grangier, G. Roger: Experimental Tests of Realistic Local Theories via Bell's Theorem. Physical Review Letters 47, 460 (1981). \url{http://journals.aps.org/prl/abstract/10.1103/PhysRevLett.47.460}.

Yao, A.: Probabilistic computations: Toward a unified measure of complexity. Proceedings of the 18th IEEE Symposium on Foundations of Computer Science (FOCS), pp. 222-227 (1977).
