\documentclass[12pt]{article}
\usepackage[margin=1in]{geometry}
\usepackage{amsmath, amssymb, amsthm, amsfonts}
\usepackage{mathdots}
\usepackage{array}
\usepackage{bbold}
\usepackage{stmaryrd}
\usepackage{setspace}
\usepackage{float}
\usepackage{pbox}
\usepackage{enumitem}
\usepackage{longtable}
\usepackage{multicol}
\usepackage{multirow}
\usepackage{booktabs}
\usepackage{titlesec}
\usepackage{refcount}
\usepackage{xspace}
\usepackage{xcolor}
\usepackage{graphicx}
\usepackage{calc}
\usepackage{fp}
\usepackage{tikz}
\usepackage{tikz-qtree}
\usepackage{forest}
\usepackage{xstring}
\usepackage[ruled,noline,noend]{algorithm2e}
\usepackage{caption}
\usepackage{subcaption}
\usepackage{adjustbox}
\usepackage{tabu}
\usepackage{arydshln}
\usepackage{xparse}
\usepackage{suffix}
\usepackage{listings}
\usepackage{verbatim}
\usepackage{pgfplots}
\usepackage{forloop}
\usepackage{braket}
\usepackage{dsfont}
\usepackage{fancyvrb}
\usepackage[T1]{fontenc}

\newtheorem{theorem}{Theorem}

\title{A Survey of Yao's Minimax Principle} % can be changed to whatever
\author{Brian Shimanuki \footnote{bshimanuki@mit.edu} \\
Mayuri Sridhar \footnote{mayuri@mit.edu} \\
Chelsea Voss \footnote{csvoss@mit.edu}}

\begin{document}

\maketitle

\begin{abstract}

\end{abstract}

\newcommand{\Expected}{\displaystyle\mathop{\mathds{E}}}
\newcommand{\argmin}{\displaystyle\mathop{\text{arg\;min}}}

\section{Introduction}

Yao's minimax principle, proposed by Yao in 1977, was first used to study lower bounds on the optimal running times of randomized algorithms for a particular problem. In this paper, we survey a series of games for which Yao's principle can be used to bound the effectiveness of all randomized algorithms against an adversary.
In this section, we present Yao's Principle and give a proof of it.

But first, we clarify the definition of a randomized algorithm.

\begin{description}
	\item[Randomized Algorithm] An algorithm which has the capabilities of a deterministic algorithm but with additional access to a stream of uniformly random bits.
\end{description}

Thus we can model a randomized algorithm as a deterministic algorithm with an additional secondary input of a stream of random bits. Conditioned on this stream, this is an entirely deterministic algorithm. Thus a randomized algorithm is modeled by a set of deterministic algorithms from which the randomized algorithm chooses at runtime with some probability distribution, before seeing the primary input.

%\subsection{Yao's Principle}

\begin{description}
	\item[Yao's Minimax Principle] \emph{The best deterministic algorithm for any particular distribution of inputs runs at least as well as any randomized algorithm when run on its own worst-case input}.
\end{description}
More formally,

$$\min_{r \in \mathcal{R}} \Expected_{x \in X} \texttt{cost}(r, x) \leq \max_{x \in \mathcal{X}} \Expected_{r \in R} \texttt{cost}(r, x)$$

where \begin{align*}
\mathcal{X} =&\; \text{the set of all possible inputs to the algorithm.}
\\
\mathcal{R} =&\; \text{the set of all possible strings defining deterministic algorithms.}
\\
X =&\; \text{any probability distribution over inputs to the algorithm.}
\\
R =&\; \text{\parbox[t]{5.5in}{any randomized algorithm, ie., any probability distribution over strings defining deterministic algorithms.}}
\\
\texttt{cost}(r, x) =&\; \text{\parbox[t]{5.5in}{the cost (often running time but in this paper will be an optimization quantity) of the algorithm $r$ running on input $x$.}}
\end{align*}

In other words, if you can figure out how well the best deterministic algorithm runs against any distribution $X$ of random inputs, you know that any randomized algorithm you formulate can do no better against its worst-case input -- giving you a lower bound on the randomized algorithm's running time.
Although $X$ need not be the worst possible distribution over random inputs for this principle to be useful, note that the worse $X$ is as an input distribution, the tighter the resulting lower bound will be. 

To further develop the intuition behind Yao's minimax principle, we will briefly cover its proof.

\subsection{Direct Proof of Yao's Principle}

Consider an arbitrary input distribution $X$. Let $d_X$ be the best deterministic algorithm for input distribution $X$, that is, the deterministic algorithm which minimizes expected cost $\Expected_{x\in X}\texttt{cost}(d_X,x)$.

Now since any randomized algorithm $R$ can be modeled as a distribution of deterministic algorithms, the expected cost of $R$ on $X$ is $\Expected_{x\in X}\Expected_{r\in R}\texttt{cost}(r,x)$, or $\Expected_{r\in R}\Expected_{x\in X}\texttt{cost}(r,x)$ by linearity of expectation. This is the weighted average of the expected costs of the deterministic algorithms in the distribution of $R$. Since the cost of each of these is at least $\Expected_{x\in X}\texttt{cost}(d_X,x)$, we have
\[\Expected_{r\in R}\Expected_{x\in X}\texttt{cost}(r,x) \ge \Expected_{x\in X}\texttt{cost}(d_X,x)\]

The cost of $R$ on its worst-case input cannot be greater than the cost of $R$ on this input distribution. Then
\[\max_{x\in\mathcal{X}}\Expected_{r\in R}\texttt{cost}(r,x)
\ge \Expected_{x\in X}\Expected_{r\in R}\texttt{cost}(r,x)
= \Expected_{r\in R}\Expected_{x\in X}\texttt{cost}(r,x)
\ge \Expected_{x\in X}\texttt{cost}(d_X,x)\]

Now $d_X$ is simply $\argmin_{r\in\mathcal{R}}\Expected_{x\in X}\texttt{cost}(r,x)$ by definition. Thus we have
\[\max_{x\in\mathcal{X}}\Expected_{r\in R}\texttt{cost}(r,x) \ge \min_{r\in\mathcal{R}}\Expected_{x\in X}\texttt{cost}(r,x)\]

\subsection{Proof of Yao's Principle}

Consider a two-player game, as defined in game theory. Each player $i$ has a set of moves $M_i$ that they may choose from. The game is defined by a \emph{payoff matrix} of outcomes for each player, a function mapping $M_1 \times M_2 \rightarrow \mathds{R} \times \mathds{R}$.

In a \emph{pure strategy}, player $1$ chooses exactly one move $x$ from $M_1$ to play. In a \emph{mixed strategy}, player $1$ instead chooses a probability distribution $\pi_1(x)$ of moves from $M_1$.

Let $\texttt{payoff}_1(x, y)$ denote the payoff to player 1 when the players choose moves $x$ and $y$, respectively.

%% \textbf{Lemma:} If you ``announce'' your mixed strategy -- thus committing to not change it -- then there exists a pure strategy for me that does at least as good as any mixed strategy for me. That is:

%% $$\forall \pi_1, \pi_2, \exists x_{opt}, \Expected_{y \in \pi_2} \texttt{payoff}_1(x_{opt}, y) \geq \Expected_{y \in \pi_2, x \in \pi_1} \texttt{payoff}_1(x, y)$$

%% \begin{proof}

%% Each component move $x_i$ of $\pi_1$ contributes a component $\pi_1(x_i) \texttt{payoff}_1(x_i, y)$ to the sum $\Expected_{x \in \pi_1} \texttt{payoff}_1(x, y)$. There exists some move $x_{opt}$ for which $\texttt{payoff}_1(x_i, y)$ is maximized. If we choose that $x_{opt}$ as a pure strategy instead, then $\Expected_{y \in \pi_2} \texttt{payoff}_1(x_{opt}, y) \geq \Expected_{y \in \pi_2, x \in \pi_1} \texttt{payoff}_1(x, y)$ as desired.

%% \end{proof}

This leads to von Neumann's minimax theorem,

$$\max_{A \in \mathcal{A}} \min_{B \in \mathcal{B}} \Expected_{a \in A, b \in B} \texttt{payoff}_1(a, b) = \min_{B \in \mathcal{B}} \max_{A \in \mathcal{A}} \Expected_{a \in A, b \in B} \texttt{payoff}_1(a, b)$$


where \begin{align*}
\mathcal{A} =&\; \text{set of moves available to player 1.}
\\
\mathcal{B} =&\; \text{set of moves available to player 2.}
\\
A =&\; \text{any probability distribution over moves to player 1}
    \\&\text{-- that is, any mixed strategy for player 1.}
\\
B =&\; \text{any probability distribution over moves to player 2}
    \\&\text{ -- that is, any mixed strategy for player 2.}
\\
\texttt{payoff}_1(a, b) =&\; \text{the payoff to player 1}
\end{align*}

This in turn can be rewritten to yield the following:

$$\max_{a \in \mathcal{A}} \Expected_{b \in B} \texttt{payoff}_1(a, b) \geq \max_{b \in \mathcal{B}} \Expected_{a \in A} \texttt{payoff}_1(a, b)$$

for any mixed strategies $A$ and $B$.

Finally, given this inequality, we need only observe the following correspondence between algorithms and strategies and between inputs and strategies in order to yield Yao's principle, as follows:

$$\min_{r \in \mathcal{R}} \Expected_{x \in X} \texttt{cost}(r, x) \leq \max_{x \in \mathcal{X}} \Expected_{r \in R} \texttt{cost}(r, x)$$

for any distribution $X$ of inputs and any distribution $R$ of randomized algorithms.

\begin{itemize}

\item{The set of moves for player 1 ($\mathcal{A}$) is the same as the set of deterministic algorithms ($\mathcal{R}$).}
\item{The set of moves for player 2 ($\mathcal{B}$) is the same as the set of possible inputs ($\mathcal{X}$).}
\item{A mixed strategy for player 1 ($A$) is the same as a randomized algorithm -- that is, a probability distribution over deterministic algorithms ($R$).}
\item{A mixed strategy for player 2 ($B$) is the same as a probability distribution over inputs to the algorithm ($X$).}
\item{The payoff to player 1 ($\texttt{payoff}$) is the negated cost to player 1 ($-\texttt{cost}$).}

\end{itemize}


%% The following table summarizes each of the parts of Yao's principle, and their roles. As we demonstrate examples which utilize Yao's principle, this table will be used in order to clarify how the principle applies to each new situation.

%% \begin{tabular}{r l} %% We can definitely change this table. This is just a first draft.
%% Randomized algorithm chooses: & some distribution $R$ over strings $r$ as its source of randomness
%% \\
%% Deterministic algorithm chooses: & some fixed string $r$
%% \\
%% Mixed-strategy adversary chooses: & some distribution $X$ over inputs $x$ for the algorithm
%% \\
%% Pure-strategy adversary chooses: & some fixed input $x$
%% \\
%% Adversary wants to maximize: & the running time of the algorithm, $\texttt{cost}(r,x)$
%% \end{tabular}

% Add more for other sections

\section{Removable Online Knapsack Problem}
In this section, we present the 2-competitive solution of Han et al.~\cite{han} to the removable online knapsack problem and use Yao's principle to show $1+1/e$ is a lower bound on the competitiveness of any algorithm against an oblivious adversary.

The \textbf{Knapsack Problem} asks: Given a set of items $e_i$ with weights $w_i$ and values $v_i$, select a subset with the maximum sum of values such that the sum of the weights is at most the capacity of the knapsack. We call this sum of values the \emph{value} of the knapsack. The offline version is a classic NP-hard optimization problem, though it has a simple dynamic programming pseudo-polynomial solution. Here, we consider the \emph{online} version, meaning each item is given to the algorithm after the algorithm makes a decision on the previous item. \emph{Removable} means the knapsack is a working set, from which we can remove items and add new items, but not add previously removed items. At all times, the knapsack is limited by its capacity. The objective is to maximize the value of the final knapsack. We say an algorithm is $\alpha$-competitive if the expected value of the final knapsack is at least $\frac{1}{\alpha}$ times the expected value of the optimal knapsack.

We assume the knapsack has capacity 1. This is valid because only weights relative to the knapsack's capacity matters for the capacity constraint, so we can always scale down to a unit capacity.

\subsection{Upper Bound} % Probably will shorten this subsection
To prove an upper bound on the competitiveness of algorithms which yield a solution to the removable online knapsack problem, we present a 2-competitive randomized algorithm.

Since the online nature of the problem is what makes it impossible to always achieve the optimal solution, it is insightful to first consider algorithms which behave similarly in both online and offline situations. We look at $MAX$ and $GREEDY$.

$MAX$ simply selects the item with the largest value whose weight does not exceed 1. In the offline case, we can process each item in an arbitrary order and keep track of the best item so far. Therefore, in the online case, we can do exactly the same with the order given to us, storing the best item so far in our knapsack.

$GREEDY$ selects the $k$ items with the highest ratios of value to weight $\frac{v_i}{w_i}$, maximizing $k$ while keeping the sum of the weights of the selected items at most 1. In the offline case, this can be done trivially by sorting the items by ratio and then selecting the highest $k$ items. An online algorithm can perform at least as well by keeping a working set of the items with the highest ratio while processing each input.
%We demonstrate that a set at least as good is found in the online case by the following algorithm.

\begin{algorithm}
	\caption{online $GREEDY$}
	$S \leftarrow \emptyset$\;
	\ForEach{item $e_i$ with $w_i\le1$, in order of arrival}{
		$S \leftarrow S \cup \{e_i\}$\;
		\While{$\sum_{e_j \in S} w_j > 1$}{
			Remove the item from $S$ with the smallest ratio\;
		}
	}
\end{algorithm}

%We claim every item selected in the offline $GREEDY$ algorithm is selected in the online $GREEDY$ algorithm. Consider the $k$ items the offline algorithm selected. Each
When the sum of the weights in $S$ is greater than 1, an item with ratio less than that of the $k$ items with the largest ratios must be in $S$; therefore the $k$ items will never be removed. Thus $S$ is a superset of the items the offline $GREEDY$ algorithm selects.

Neither $MAX$ nor $GREEDY$ alone are competitive. Consider an input sequence $\{(w_i,v_i)\}$ of $\{(\epsilon,2\epsilon),\ (\epsilon,\epsilon),\ (\epsilon,\epsilon),\ \ldots\}$ on the $MAX$ algorithm and $\{(\epsilon, 2\epsilon),\ (1,1)\}$ on the $GREEDY$ algorithm. In fact, no deterministic online algorithm is competitive with a constant factor \cite{iwama}. However, using both with randomness can yield a competitive algorithm.

\begin{theorem}
	\emph{\cite{han}}
	Running $MAX$ and $GREEDY$ uniformly at random is a 2-competitive algorithm.
\end{theorem}
\begin{proof}
	For a given input sequence $T$, let $GREEDY(T)$, $MAX(T)$, and $OPT(T)$ be the values of the sets returned by $GREEDY$, $MAX$, and an optimal solution, respectively. Consider the \emph{fractional} knapsack solution to the offline problem with input sequence $T$: that is, when we are allowed to take fractional parts of items. Then, a greedy solution is optimal: Sort by ratio, and take the highest ratio items until the knapsack is full, potentially taking only part of the last item in the knapsack. The fractional knapsack problem is a relaxation of the integral knapsack problem, so the fractional solution is not worse than the integral solution.
	
	Now the offline $GREEDY$ algorithm by definition selects all the items in the fractional solution except the final fractional item (if one exists). Thus the online $GREEDY$ algorithm also selects all except the fractional item. Finally, the $MAX$ algorithm selects an item with value at least that of the fractional item, so $GREEDY(T)+MAX(T) \ge OPT(T)$. Thus the competitive ratio is $\frac{2\cdot OPT(T)}{GREEDY(T)+MAX(T)} \le 2$.
\end{proof}

\subsection{Lower Bound}
We use Yao's principle to prove a lower bound of $1+1/e$ for the competitive ratio of any removable online knapsack algorithm.

To do so, we must find a distribution of inputs for which no deterministic algorithm performs ``well'', which means the expected value of the output from any deterministic algorithm is at most $\frac{1}{1+1/e}$ times the expected optimum. As the power of an offline algorithm is in the additional information it has over an online algorithm, it is reasonable that in constructing our ``bad'' input, we would want to limit the information the algorithm has about future items. Thus it is not surprising that we will construct a family of inputs where each input is a prefix of the same base sequence.

Let the probability of $k+1$ items be $p_k$. For simplicity in analysis, we will let the first item $e_0=(w_0,v_0)=(1,1)$ with future items having much smaller weights and values but a higher ratio so that the problem is essentially reduced to determining when to remove the first item from the knapsack.

This is reminiscent of the \emph{ski rental problem}, one of the most well-understood online problems \cite{karlin}, where a client must choose whether and when to buy skis for a fixed cost versus paying a smaller cost each day. Our knapsack problem is the reverse of this: instead of deciding when to switch from renting skis to buying skis to minimize total cost, we are deciding when to switch from having a single large item in our knapsack to getting a stream of smaller items with a better ratio to maximize total value.

Finally, we want the expected value of removing $e_0$ and selecting all following items to be close to 1 at any given point. A simple way to achieve this is to let the remaining items have a constant value and have the $p_k$ form an exponential distribution.

This leads us towards the input sequence given by Han et al. \cite{han}:
\begin{equation}
	\label{eq:knapsack_dist}
	(1,1),\ \underbrace{(1/n^2,1/n),\ (1/n^2,1/n),\ldots\ (1/n^2,1/n)}_{k \text{ terms}}
\end{equation}
for a given value of $n$ and where there are $k+1$ items with probability $p_k = \frac{1-e^{-1/n}}{1-e^{-n}} \cdot e^{-(k-1)/n}$ for $k=1,2,\ldots,n^2$.

\begin{theorem}
	\emph{\cite{han}}
	No online algorithm has competitive ratio less than $1+1/e$ for the removable online knapsack problem.
\end{theorem}
\begin{proof}
	We apply Yao's principle on the input distribution \eqref{eq:knapsack_dist}. The optimal strategy with full information is to keep $e_0$ for $k\le n$ and remove $e_0$ immediately otherwise. This leads us to the expected optimal value
	\[ \sum_{i=1}^n 1\cdot p_i + \sum_{i=n+1}^{n^2} \frac{i}{n}\cdot p_i = \frac{1-e^{-1/n}}{1-e^{-n}} \left( \sum_{i=1}^n e^{-(i-1)/n} + \frac{1}{n} \sum_{i=n+1}^{n^2} i\cdot e^{-(i-1)/n} \right) \]
	which has value $1+1/e$ as $n \rightarrow \infty$.

	Since on this set of inputs, the initial item has weight 1 and the sum of the rest of the items is at most 1, any optimal algorithm having removed the initial item can perform a greedy algorithm and put all future items in the knapsack. Thus an optimal online deterministic algorithm can only decide how many items it must see before it removes $e_0$. For inputs $k<l$, the algorithm keeps $e_0$ and removes all incoming items, so it obtains value 1. For $k\ge l$, the algorithm removes $e_0$ when the $l$\textsuperscript{th} item appears, and accumulates items $e_l$ through $e_k$. Then the expected value of this algorithm is
	\[ \sum_{i=1}^{l-1} 1\cdot p_i + \sum_{i=l}^{n^2} \frac{i-l}{n}\cdot p_i = \frac{1-e^{-1/n}}{1-e^{-n}} \left( \sum_{i=1}^{l-1} e^{-(i-1)/n} + \frac{1}{n} \sum_{i=l}^{n^2} (i-1)\cdot e^{-(i-1)/n} \right) \]
which has value 1 as $n \rightarrow \infty$.

By Yao's principle, since every deterministic algorithm finds $(1+1/e)^{-1}$ of the optimum set on this distribution of inputs, any online algorithm performs at least as poorly. Thus any online algorithm to this problem has competitive ratio at least $1+1/e$.
\end{proof}

%\subsection{Remarks}
%Yao's Principle is used here to give a lower bound on the competitiveness of algorithms for the removable online knapsack problem, which is an upper bound on their effectiveness. It can be used to show the tightness of a particular solution.

%Also, as we saw here, there is a dual between using an algorithm to demonstrate a lower bound on the solvability of a problem (alternatively, an upper bound on the hardness of a problem), and using a series of inputs to demonstrate an upper bound on the solvability of a problem (alternatively, an lower bound on the hardness). As more work is done on a particular problem, the gap between the two shrink until a provably optimal algorithm is found.

\documentclass{article}
\usepackage[utf8]{inputenc}
\usepackage{fancyvrb}
\usepackage{bera}
\usepackage{amsmath}

\title{6.856 Final Project}
\author{ }
\date{May 2015}

\begin{document}

\maketitle

\section{Online Coloring Co-Interval Graphs}
This section studies the online problem of coloring co-interval graphs. \\This problem is set up as follows:
\\Input: A set of intervals on the real line in arbitrary order
\\Output: A color matched to each interval satisfying the condition that any interval $I_k$ must be assigned a color that's different from the color assigned to any $I_j$ such that $j < k$ and intervals $j$ and $k$ don't intersect.
\\Since this is an online problem, the algorithm must assign colors as it goes, without knowing the complete set of intervals. That is, the algorithm will only receive information about $I_{j+1}$ after it has assigned colors to all intervals $I_k$ where $k \leq j$.
\\First, we present the First-Fit approximation algorithm to solve this problem, which has a competitive ratio of at most 2. Then, using Yao's Principle, we can prove that there exist no randomized algorithm that achieves a competitive ratio that's better than $\frac{3}{2}$.
\subsection{Upper Bound}
To solve the problem of assigning intervals different colors, the set of intervals is first constructed as a graph. That is, the input to our problem becomes:
\\$V$ ::== Set of intervals where $v_i$ represents the $i$'th interval
\\$E$ ::== Set of edges where $(u,v) \in E$ iff interval $u$ and interval $v$ are disjoint
\\Note that any vertices that aren't connected by an edge can be colored in the same color since they overlap. Thus, in such a graph, our problem is reduced to finding the minimum number of colors such that no connected vertices are the same color.
\\The First-Fit algorithm is the simplest solution to this problem and does the following:
\begin{verbatim}
L = sorted list of colors
for a new vertex v:
    assign v the smallest color that doesn't cause a collision
\end{verbatim}
Essentially, the algorithm maintains a list of colors that can be used. These colors are sorted in an arbitrary order. The first vertex is simply assigned the smallest color. For any future vertex $v$, the algorithm scans the list. We can define a collision between vertices $u$ and $v$ at color $k$ if $u$ is colored $k$ and $v$ is connected to $u$ by an edge in $E$. If a collision occurs where $L[i] = k$, then we increment $i$. We assign $v$ the color at $L[j]$ where $j$ doesn't cause any collisions and try to minimize $j$.
\\It has been shown by Gyarfas and Lehel that First-Fit uses at most $2\omega+1$ colors on any co-chordal graph, which is a class that includes co-interval graphs. It has also been shown by Kierstead and Qin that no online deterministic algorithm can beat the 2-competitive ratio on co-interval graphs.
\\In the next section, we look at using randomization in co-interval graphs and lower bound the best competitive ratio that can be obtained.
\subsection{Lower Bound}
Yao's minimax principle proves that the expected competitive ratio on an optimal deterministic algorithm on the worst-case input is a lower bound on the expected competitive ratio of all randomized algorithms. That is, if there exists a set of inputs such that all optimal deterministic algorithms are at least $\alpha$-competitive, then $\alpha$ is the best ratio that can be obtained by using randomization.
\\To define such a set of inputs, essentially, we need to find a set of intervals that satisfy the condition that any deterministic algorithm is at most $\frac{3}{2}$-competitive. The idea behind constructing such a set of intervals is that any deterministic algorithm will make the same decision on any specific interval given the same information, regardless of what the future intervals are.
\\The intervals defined by Zarrabi-Zadeh are: 
\begin{Verbatim}[commandchars=\\\{\},codes={\catcode`$=3\catcode`_=8}]
For k $\geq$ 1:
    For all 1 $\leq i \leq k$:
        $a_i$ = [3$i$-3, 3$i$-2]
        $b_i$ = [3$i$+1, 3$i$+2]
        $c_i$ = [3$i$+2, +$\infty$]
\end{Verbatim}
We can then define a block as:
$B_i$ = \begin{cases} [$a_1, $b_1$] &\mbox{if } i \equiv 1 \\ 
[$a_i$, $b_i$, $c_i$] & \mbox{if } i \in [2,k] \\ [$a_k$, $b_k$] &\mbox{if } i \equiv k \end{cases}
\\From this set of blocks, we can choose a set of $k$ input sequences, simply by defining our input sequence $X_i$ as the concatenation of blocks $B_1$, $B_2$... $B_i$, in order.
\\It can be shown that the minimum number of colors required to color $X_i$ without any edges sharing a color is $i$.
\\To use Yao's Principle on this input distribution, we need to show that if we set up a probability distribution over all possible $X_i$ that the expected competitive ratio will be at least $\frac{3}{2}$. The optimal deterministic algorithm that we will examine per each $X_i$ input is First-Fit - as previously shown, this algorithm uses $2\omega-1$ colors on each $X_i$. In this case, the chromatic number $\omega=i$, so, the expected number of colors for a given $X_i$ is exactly $2i-1$.

\subsection{Remarks}

\end{document}

\newcommand{\Mid}{\nobreak\mid\nobreak}

\section{Yao's Principle in Quantum Information}

Yao's principle isn't just used for proving lower bounds on the competitiveness of algorithms. Another application is found in the study of certain games in quantum information, where Yao's principle can show worst-case bounds of game strategies. In this section, we will examine this use, as demonstrated in \emph{Worst Case Analysis of Non-local Games} by Ambainis et al.~\cite{ABBSSV}.

\subsection{Quantum Preliminaries}

Consider a game between two players and an adversary, where the two players receive random input from the adversary and must each return outputs jointly satisfying a condition without communicating with each other during the game. Analysis under classical physics conditions and quantum physics conditions has demonstrated quantum information allows for better strategies.

An example of such a game is the Clauser--Horne--Shimonyi--Holt (CHSH) game~\cite{CHSH}:

\begin{itemize}
\item{The two players may collude to set up a strategy beforehand but may not communicate afterwards.}
\item{The adversary chooses a random $\vec{x}=(x_1, x_2)$ from~$X=\{(0,0), (0,1), (1,0), (1,1)\}$. The random choice is uniform, with $\frac14$ probability each.}
\item{The players 1~and~2 receive the values $x_1$~and~$x_2$, respectively.}
\item{In the quantum setting, the players 1~and~2 may also possess an entangled 2-part quantum state~$\ket{\psi}$, which they may measure.}
\item{Using their strategy, the players 1~and~2 output the values $a_1(x_1)$~and~$a_2(x_2)$, respectively.}
\item{The players win if $a_1 \oplus a_2 = x_1 \wedge x_2$.}
\end{itemize}

Both under classical physics and under any ``local hidden variable theory'' -- an explanation of quantum phenomena as being due to extra local hidden variables -- the best that the players can achieve is a win probability of~$0.75$. If the players are allowed to share entangled quantum bits, however, then they can win more often, with a win probability of~$\frac12 + \frac1{2\sqrt2} = 0.8535\ldots$~\cite{ABBSSV,CHSH}.

The CHSH game was designed as a testable experiment, and experiments have since confirmed that the universe is consistent with quantum entanglement, and not consistent with local hidden variable theories~\cite{aspect}.

In \emph{Worst Case Analysis of Non-local Games} by Ambainis et al., the authors examine generalized games beyond the simple CHSH game, and analyze the worst-case performance of strategies in both classical and quantum environments~\cite{ABBSSV}.

\subsection{Generalized CHSH}

The CHSH game is played between two players and an adversary. We generalize the game from two players to $n$ players. Ambainis et al.~call this the \emph{$n$-party AND game}~\cite{ABBSSV}.

The authors note that this game has not been studied before because of the simplicity of its strategies in the classical setting: by outputting $a_i=0$, the players are guaranteed to win unless the adversary happened to pick $x_i=1$. Therefore, against an adversary that always chooses $\vec{x}$ uniformly, the players can win with probability $1-2^n$.

However, this game becomes more interesting when we consider worst-case bounds: if the adversary is allowed to choose $\pi$, the optimal strategy for the players becomes less trivial. The authors show that $\lim_{n \rightarrow \infty}\Pr[win] = 2/3$ in both the classical case and the quantum case~\cite{ABBSSV}.

The $n$-party AND game and the CHSH game share the same win condition, $\bigoplus \vec{a} = \bigwedge \vec{x}$. To generalize this further, we can imagine games with arbitrary win conditions. Let the \textbf{Generalized CHSH Game} be as follows.

%% \begin{itemize}
%% \item{The $n$ players may collude to set up a strategy beforehand, but may not communicate afterwards.}
%% \item{The adversary chooses a random $\vec{x}=(x_1, ..., x_n) \in X_1 \times ... \times X_n$, for some $X_1, ..., X_n$. The random choice is governed by a probability distribution $\pi$; each $\vec{x}$ is chosen with probability $\pi(\vec{x})$.}
%% \item{The players $1, ..., n$ receive the values $x_1, ..., x_n$, respectively.}
%% \item{In the quantum setting, the players may also possess an entangled $n$-part quantum state $\ket{\psi}$, which they may measure.}
%% \item{Using their strategy, the players $1, ..., n$ output the values $\vec{a} = a_1(x_1), ..., a_n(x_n)$, respectively.}
%% \item{The players win if $\bigoplus \vec{a} = \bigwedge \vec{x}$. Let the indicator variable $\texttt{win}(\vec{a}|\vec{x}) = 1$ if they win, $-1$ otherwise.}
%% \end{itemize}

%%

%%

%% TODO: transition here

%% \textbf{Theorem:} TODO

%% \begin{proof}
%% In previous results [[cite Ambainis 2010]], the authors demonstrate that the value of [...TODO].
%% \end{proof}

% \subsection{Fully Generalized Games}



\begin{itemize}
\item{The $n$ players may collude and strategize beforehand but may not communicate afterwards.}
\item{The adversary chooses a random $\vec{x}=(x_1, \ldots, x_n)$ from~$X_1 \times \cdots \times X_n$, for some $X_1, \ldots, X_n$. The random choice is governed by a probability distribution~$\pi$; each $\vec{x}$ is chosen with probability~$\pi(\vec{x})$.}
\item{The players~$1, \ldots, n$ receive the values~$x_1, \ldots, x_n$, respectively.}
\item{In the quantum setting, the players may also possess an entangled $n$-part quantum state~$\ket{\psi}$, which they may measure.}
\item{Using their strategy, the players~$1, \ldots, n$ output the values~$\vec{a} = a_1, \ldots, a_n$, respectively.}
\item{The players win if some proposition~$\texttt{win}(\vec{a} \Mid \vec{x})$ is true. Let the indicator variable~$\texttt{win}(\vec{a} \Mid \vec{x}) = 1$ if they win and $-1$ otherwise.}
\end{itemize}

For any game~$G$, define $\omega^\pi(G)$ as $\Pr[\textrm{Win}] - \Pr[\textrm{Lose}]$ when $\pi$ is a probability distribution according to which the adversary chooses $\vec{x}$. Let $\omega(G)$ be the worst case $\omega^\pi(G)$: the minimum value of $\omega^\pi(G)$ over distributions $\pi$ that the adversary can choose from. Let $\omega_q$ and $\omega_c$ be $\omega$ in the quantum and classical cases, respectively. We defined the indicator variable $\texttt{win}(\vec{a}\Mid\vec{x})$ such that $\Pr[\textrm{Win}] - \Pr[\textrm{Lose}] = \Expected[ \texttt{win}(\vec{a}\Mid\vec{x}) ]$. Therefore, we can more formally define $\omega(G)$ and $\omega^\pi(G)$ as follows:
\begin{align*}
	\omega^\pi(G) &\equiv \max_{\vec{a}} \Expected_{\vec{x} \in \pi} \texttt{win}(\vec{a}\Mid\vec{x}) \\
	\omega(G) &\equiv \max_{\vec{a}} \min_\pi \Expected_{\vec{x} \in \pi} \texttt{win}(\vec{a}\Mid\vec{x}).
\end{align*}

Recent research in quantum information has considered finding games which exemplify a difference between the quantum and classical settings: seeking to maximize the ratio~$\omega_q^\pi(G) / \omega_c^\pi(G)$~\cite{ABBSSV}. Interestingly, for every possible proposition $\texttt{win}$, the separation ratio~$\omega_q(G) / \omega_c(G)$ when the adversary chooses the worst distribution~$\pi$ \emph{separately} for the classical and quantum case is at most the maximum value of $\omega_q^\pi(G) / \omega_c^\pi(G)$ over a fixed distribution~$\pi$. This proof uses Yao's minimax principle.

\begin{theorem}
	$$\frac{\omega_q(G)}{\omega_c(G)} \leq \max_\pi \frac{\omega_q^\pi(G)}{\omega_c^\pi(G)}.$$
\end{theorem}

\begin{proof}

As with the earlier applications of Yao's minimax principle, the adversary chooses a strategy over a distribution of inputs~$\vec{x}$ to pass to the players, and the players may still use randomized algorithms as strategies. However, instead of letting the cost be \emph{running time}, the cost is the \emph{probability of winning}. The adversary attempts to minimize cost and the two players' joint algorithms attempt to maximize cost.

Recall Yao's minimax principle, but instead of stating that any (including the best) randomized algorithm when run on its own worst-case input can perform no better than the best deterministic algorithm when run on any (including the worst) specific distribution of inputs, we will now state that the players' best randomized algorithms when run on their worst-case input can perform no better than the players' best deterministic algorithms when run on the worst distribution~$\pi$ of inputs from the adversary.

The players' best deterministic algorithm against the worst distribution~$\pi$ over random inputs from the adversary is $\max_{\vec{a}} \min_{\pi} \Expected_{\vec{x} \in pi} \texttt{win}(\vec{a}\Mid\vec{x})$, which is defined to be $\omega_c(G)$. The players' best randomized algorithm against the worst input is $\min_\pi \max_{\vec{a}} \Expected_{\vec{x} \in pi} \texttt{win}(\vec{a}\Mid\vec{x})$, which is $\min_\pi \omega^\pi_c(G)$.

This gives us a bound on the worst-case win probability under classical conditions: $\omega_c(G) \geq \min_{\pi} \omega_c^\pi(G)$. Suppose $\pi$ is the probability distribution that achieves this minimum; then $\omega_q^\pi(G) \geq \omega_q(G)$, because fixing $\pi$ can only allow the players to achieve a better win probability than their worst-case $\omega_q(G)$.
\end{proof}

%TODO: the following citations

%Ambainis, A., A. Ba\v{c}kurs, K. Balodis, A. \v{S}ku\v{s}kovniks, J. Smotrovs, M. Virza: Worst Case Analysis of Non-local Games. P. van Emde Boas et al. (Eds.): SOFSEM 2013, LNCS 7741, pp. 121-132, 2013. \url{http://arxiv.org/abs/1112.2856}

%Ambainis, A., Kravchenko, D., Nahimovs, N., Rivosh, A.: Nonlocal Quantum XOR Games for Large Number of Players. In: Kratochv\a'il, J., Li, A., Fiala, J., Kolman, P. (eds.) TAMC 2010. LNCS, vol. 6108, pp. 72-83. Springer, Heidelberg (2010).

%Clauser, J., M. Horne, A. Shimony, R. Holt: Proposed experiment to test local
%hidden-variable theories. Physical Review Letters 23, 880 (1969). \url{http://journals.aps.org/prl/abstract/10.1103/PhysRevLett.23.880}

%Aspect, A., P. Grangier, G. Roger: Experimental Tests of Realistic Local Theories via Bell's Theorem. Physical Review Letters 47, 460 (1981). \url{http://journals.aps.org/prl/abstract/10.1103/PhysRevLett.47.460}.

%Yao, A.: Probabilistic computations: Toward a unified measure of complexity. Proceedings of the 18th IEEE Symposium on Foundations of Computer Science (FOCS), pp. 222-227 (1977).


\section{Conclusion}
We have demonstrated Yao's principle to give lower bounds on the competitiveness of algorithms for a couple of online games, which is an upper bound on their effectiveness. When these bounds match the competitiveness of a particular algorithm, the principle can be used to show the tightness of the algorithm.

Also, there is a dual between using an algorithm to demonstrate a lower bound on the solvability of a problem (alternatively, an upper bound on the hardness of a problem), and using a series of inputs to demonstrate an upper bound on the solvability of a problem (alternatively, an lower bound on the hardness). As more work is done on a particular problem, the gap between the two shrink until a provably optimal algorithm is found.

Yao's principle is applicable to many types of costs when analyzing problems, including running time, and, as we saw here, competitiveness of algorithms in partial information scenarios.


\bibliographystyle{plain}
%\nocite{*}
\bibliography{citations}

\end{document}
